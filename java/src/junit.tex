\begin{defnbox}\nospacing
  \begin{defn}[Unit Test]\label{defn:unitTest}
     Is a level of software testing where individual units/ components of a
     software are tested i.e.\ we do not run \javainline{main} but test
     individual pieces of software.
  \end{defn}
\end{defnbox}
\begin{sectionbox}\nospacing
 Is a framework for running unit tests 
 \begin{figure}[H]
   \centering
   \begin{javacode}
     import org.junit.Test
     import static org.junit.Assert.*;

     public class MyTest{

       @Test
       public void |\optc{testName1}|(){
         
       }
       @Test
       public void |\optc{testName2}|(){
         
       }
     }
   \end{javacode}
   \caption{Basic JunitTest}
 \end{figure}
\end{sectionbox}
\begin{sectionbox}[Test Class Annotations]\nospacing
  \begin{itemizenosep}
      \item \javainline{@Test}: indicates that a method contains unit tests
      \begin{mintlinebox}{java}
        @Test|\optla|expected=|\optc{IllegalException}\optra|
      \end{mintlinebox}
      \item \mintinline{java}{@Before/@After}: mark methods that are called
    before and after each unit test.\\
    \imp{Note}: can be used to set up and tear down thins needed for all tests.
      \item \mintinline{java}{@BeforeClass/@BeforeClass}: similar to
    \javainline{@Before} and \javainline{@After} but only run once per class.
  \end{itemizenosep}
\end{sectionbox}
%%% Local Variables:
%%% mode: latex
%%% TeX-master: "../Java"
%%% End:
