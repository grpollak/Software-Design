\begin{defnbox}\nospacing
  \begin{defn}[Proxy]
     Latin word meaning ``the authority to represent someone else''.
  \end{defn}
\end{defnbox}
\begin{figure}[H]	
  \centering
    \resizebox{\linewidth}{!}{\tikzset{font=\Huge}\documentclass{standalone}
\usepackage{tikz}
\usepackage{aeguill}
\begin{document}
% generated by Plantuml 1.2018.12      
\definecolor{plantucolor0000}{RGB}{254,254,206}
\definecolor{plantucolor0001}{RGB}{168,0,54}
\definecolor{plantucolor0002}{RGB}{173,209,178}
\definecolor{plantucolor0003}{RGB}{0,0,0}
\definecolor{plantucolor0004}{RGB}{180,167,229}
\definecolor{plantucolor0005}{RGB}{255,255,255}
\begin{tikzpicture}[yscale=-1
,pstyle0/.style={color=plantucolor0001,fill=plantucolor0000,line width=1.5pt}
,pstyle1/.style={color=plantucolor0001,fill=plantucolor0002,line width=1.0pt}
,pstyle2/.style={color=plantucolor0001,line width=1.5pt}
,pstyle4/.style={color=plantucolor0001,line width=1.0pt}
,pstyle5/.style={color=plantucolor0001,fill=plantucolor0001,line width=1.0pt}
,pstyle6/.style={color=plantucolor0001,fill=white,line width=1.0pt}
,pstyle7/.style={color=black,fill=black,line width=1.0pt}
]
\draw[pstyle0] (6pt,27pt) rectangle (81.7514pt,75pt);
\draw[pstyle1] (21pt,43pt) ellipse (11pt and 11pt);
\node at (21pt,43pt)[]{\textbf{\Large C}};
\node at (35pt,36.0156pt)[below right,color=black]{Client};
\draw[pstyle2] (7pt,59pt) -- (80.7514pt,59pt);
\draw[pstyle2] (7pt,67pt) -- (80.7514pt,67pt);
\draw[pstyle0] (166.5pt,8pt) rectangle (251.9213pt,94.4141pt);
\draw[color=plantucolor0001,fill=plantucolor0004,line width=1.0pt] (181.5pt,24pt) ellipse (11pt and 11pt);
\node at (181.5pt,24pt)[]{\textbf{\Large I}};
\node at (195.5pt,17.0156pt)[below right,color=black]{\textit{Subject}};
\draw[pstyle2] (167.5pt,40pt) -- (250.9213pt,40pt);
\draw[pstyle2] (167.5pt,48pt) -- (250.9213pt,48pt);
\node at (172.5pt,52pt)[below right,color=black]{request1()};
\node at (172.5pt,64.8047pt)[below right,color=black]{request2()};
\node at (172.5pt,77.6094pt)[below right,color=black]{request3()};
\draw[pstyle0] (58pt,155pt) rectangle (140.2462pt,254.2188pt);
\draw[pstyle1] (77.9451pt,171pt) ellipse (11pt and 11pt);
\node at (77.9451pt,171pt)[]{\textbf{\Large C}};
\node at (93.044pt,164.0156pt)[below right,color=black]{Proxy};
\draw[pstyle2] (59pt,187pt) -- (139.2462pt,187pt);
\node at (64pt,191pt)[below right,color=black]{realSubject};
\draw[pstyle2] (59pt,207.8047pt) -- (139.2462pt,207.8047pt);
\node at (64pt,211.8047pt)[below right,color=black]{request1()};
\node at (64pt,224.6094pt)[below right,color=black]{request2()};
\node at (64pt,237.4141pt)[below right,color=black]{request3()};
\draw[pstyle0] (225pt,161.5pt) rectangle (341.3123pt,247.9141pt);
\draw[pstyle1] (240pt,177.5pt) ellipse (11pt and 11pt);
\node at (240pt,177.5pt)[]{\textbf{\Large C}};
\node at (254pt,170.5156pt)[below right,color=black]{RealSubject};
\draw[pstyle2] (226pt,193.5pt) -- (340.3123pt,193.5pt);
\draw[pstyle2] (226pt,201.5pt) -- (340.3123pt,201.5pt);
\node at (231pt,205.5pt)[below right,color=black]{request1()};
\node at (231pt,218.3047pt)[below right,color=black]{request2()};
\node at (231pt,231.1094pt)[below right,color=black]{request3()};
\draw[pstyle4] (95.5221pt,51pt) ..controls (113.8205pt,51pt) and (134.4974pt,51pt) .. (153.1409pt,51pt);
\draw[pstyle5] (166.2938pt,51pt) -- (157.2938pt,47pt) -- (161.2938pt,51pt) -- (157.2938pt,55pt) -- (166.2938pt,51pt) -- cycle;
\draw[pstyle4] (161.2938pt,51pt) -- (153.2939pt,50.9999pt);
\draw[pstyle6] (82.269pt,51pt) -- (88.269pt,55pt) -- (94.269pt,51pt) -- (88.269pt,47pt) -- (82.269pt,51pt) -- cycle;
\node at (100.25pt,32pt)[below right,color=black]{Uses};
\draw[pstyle7] (138.5833pt,38.0664pt) -- (144.5833pt,41.0664pt) -- (138.5833pt,44.0664pt) -- (138.5833pt,38.0664pt) -- cycle;
\draw[pstyle4] (166.168pt,110.7701pt) ..controls (155.708pt,125.3665pt) and (144.63pt,140.8255pt) .. (134.5519pt,154.889pt);
\draw[pstyle4] (160.7264pt,106.3461pt) -- (178.0661pt,94.1668pt) -- (172.1062pt,114.501pt) -- (160.7264pt,106.3461pt) -- cycle;
\draw[pstyle4] (238.6191pt,112.4396pt) ..controls (246.5248pt,128.8387pt) and (254.8992pt,146.2098pt) .. (262.2181pt,161.3915pt);
\draw[pstyle4] (232.1897pt,115.2224pt) -- (229.8101pt,94.1668pt) -- (244.8008pt,109.1428pt) -- (232.1897pt,115.2224pt) -- cycle;
\draw[pstyle4] (153.394pt,204.5pt) ..controls (171.6372pt,204.5pt) and (192.2633pt,204.5pt) .. (211.6079pt,204.5pt);
\draw[pstyle5] (224.7506pt,204.5pt) -- (215.7506pt,200.5pt) -- (219.7506pt,204.5pt) -- (215.7506pt,208.5pt) -- (224.7506pt,204.5pt) -- cycle;
\draw[pstyle4] (219.7506pt,204.5pt) -- (211.7507pt,204.4999pt);
\draw[pstyle6] (140.2932pt,204.5pt) -- (146.2932pt,208.5pt) -- (152.2932pt,204.5pt) -- (146.2932pt,200.5pt) -- (140.2932pt,204.5pt) -- cycle;
\node at (158.5pt,185.5pt)[below right,color=black]{Uses};
\draw[pstyle7] (196.8333pt,191.5664pt) -- (202.8333pt,194.5664pt) -- (196.8333pt,197.5664pt) -- (196.8333pt,191.5664pt) -- cycle;
\end{tikzpicture}
\end{document}
}
\end{figure}
\begin{intentbox}[Intents]
  \begin{itemizenosep}
      \item 
      Use a surrogate or placeholder to control the access of original object
      $\Rightarrow$ protection of the original object from the outside world.
      \item Lazy initialization of resource hungry objects
  \end{itemizenosep}
\end{intentbox}
\begin{partbox}[Participants]
  \begin{itemizenosep}
\item \textbf{Proxy interface}:
An interface that is implemented by the concrete object and the proxy object
\item \textbf{Dynamic proxy}
\begin{itemize}
    \item 
Instance of a dynamic proxy class
\item
Dynamic proxy class implements a list of interfaces specified at runtime when
dynamic proxy instance is generated (not at compile time)
\end{itemize}
\item \textbf{Invocation Handler Object}
\begin{itemize}
    \item 
Each proxy instance has an associated invocation handler object which
implements the interface InvocationHandler
\item A method invocation on a proxy instance trough one of its proxy interfaces will
be dispatched to the invoke() method of the instance’s invocation handler
\end{itemize}
\item \textbf{Concrete Object}:
Object which is used in invocation handler
  \end{itemizenosep}
\end{partbox}
\todo[inline]{finish section lec 9 proxy}
%%% Local Variables:
%%% mode: latex
%%% TeX-master: "../formulary"
%%% End:
