\begin{sectionbox}\nospacing
  In Java, every variable, constant, and function (\textit{including main}) must
  be inside some class.\\
  \rdb{A Java Program}: is a class which contains a main method:
  \begin{mintlinebox}{java}
		public static void main(String[] args) { |\ldots| }
  \end{mintlinebox}
  \begin{itemizenosep}
      \item \javainline{pulic}: accessible from everywhere.
      \item \javainline{static}: defined on the class level (not bound to instances).
      \item \javainline{void}: does not return a result ($\Rightarrow$ procedure)
      \item \javainline{args}: argument, array of command-line arguments
  \end{itemizenosep}
\end{sectionbox}
\subsection{Packages}
\begin{sectionbox}\nospacing
  Programmers can define their own packages to bundle group of
  classes/interfaces, etc.\\
  Packages create a new namespace thus there won't be any name conflicts with
  similar identifiers from other packages.
\end{sectionbox}
\begin{defnbox}\nospacing
  \begin{defn}[Package Statement]
    Identifies the package that a Java program belongs to.
    The package statement should be the first line in the source file
    (there can be only one package statement per source file).\\
    \imp{Package Statement}:\hfil \javainline{package package-name;}
  \end{defn}
\end{defnbox}
\begin{sectionbox}[Compiling source files with package statements]\nospacing
  Using the \javainline{-d destination_fodler} option will create a folder with
  the given package name \textit{in the package statement} will be created in
  the given destination folder (if not existing), and place the complied source file
  into it.
  \begin{mintlinebox}{java}
		javac -d destination_folder file.java
  \end{mintlinebox}
\end{sectionbox}
\begin{sectionbox}[Using Packages]\nospacing
  \begin{itemizenosep}
      \item \javainline{package_name.identifier}: use fully qualified names.
    	\item \javainline{import package_name.*}: import the whole package.
    	\item \javainline{import package_name.identifier}: import certain identifiers.
  \end{itemizenosep}
\end{sectionbox}
\begin{stylebox}[The default package]\nospacing
  \begin{itemizenosep}
    \item 
  If a program does not include a package statement it belongs to the so called
  \rd{default package}, which is basically an default, unnamed package.\\
  When developing small or temporary applications e.g. for testing purposes, its
  ok not to include a package statement.\\
  \imp{But} in order to avoid name conflicts, all java source files belonging to
  a program should contain a package statement.
    \item \imp{Convention}: use your transposed internet domain name for
  uniqueness, if you have one.
    \item \imp{Lower Cases}: use lower case letters for packages in orde
      to avoid any conflicts with the names of classes and interfaces.
  \end{itemizenosep}
\end{stylebox}
%%% Local Variables:
%%% mode: latex
%%% TeX-master: "../formulary"
%%% End:
