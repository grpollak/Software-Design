\begin{figure}[H]	
  \centering
    \resizebox{\linewidth}{!}{\tikzset{font=\Huge}\documentclass{standalone}
\usepackage{tikz}
\usepackage{aeguill}
\begin{document}
% generated by Plantuml 1.2018.12      
\definecolor{plantucolor0000}{RGB}{254,254,206}
\definecolor{plantucolor0001}{RGB}{168,0,54}
\definecolor{plantucolor0002}{RGB}{173,209,178}
\definecolor{plantucolor0003}{RGB}{0,0,0}
\definecolor{plantucolor0004}{RGB}{169,220,223}
\begin{tikzpicture}[yscale=-1
,pstyle0/.style={color=plantucolor0001,fill=plantucolor0000,line width=1.5pt}
,pstyle1/.style={color=plantucolor0001,fill=plantucolor0002,line width=1.0pt}
,pstyle2/.style={color=plantucolor0001,line width=1.5pt}
,pstyle4/.style={color=plantucolor0001,line width=1.0pt}
,pstyle5/.style={color=plantucolor0001,fill=plantucolor0001,line width=1.0pt}
]
\draw[pstyle0] (6pt,8pt) rectangle (129.6941pt,120.0234pt);
\draw[pstyle1] (37.2424pt,24pt) ellipse (11pt and 11pt);
\node at (37.2424pt,24pt)[]{\textbf{\Large C}};
\node at (54.8518pt,17.0156pt)[below right,color=black]{Context};
\draw[pstyle2] (7pt,40pt) -- (128.6941pt,40pt);
\node at (12pt,44pt)[below right,color=black]{state};
\draw[pstyle2] (7pt,60.8047pt) -- (128.6941pt,60.8047pt);
\node at (12pt,64.8047pt)[below right,color=black]{+methodA()};
\node at (12pt,77.6094pt)[below right,color=black]{+allMethods()};
\node at (12pt,90.4141pt)[below right,color=black]{+setState(State s)};
\node at (12pt,103.2188pt)[below right,color=black]{+getstate()};
\draw[pstyle0] (165pt,27pt) rectangle (243.1419pt,100.6094pt);
\draw[color=plantucolor0001,fill=plantucolor0004,line width=1.0pt] (184.0464pt,43pt) ellipse (11pt and 11pt);
\node at (184.0464pt,43pt)[]{\textbf{\Large A}};
\node at (198.9456pt,36.0156pt)[below right,color=black]{\textit{State}};
\draw[pstyle2] (166pt,59pt) -- (242.1419pt,59pt);
\draw[pstyle2] (166pt,67pt) -- (242.1419pt,67pt);
\node at (171pt,71pt)[below right,color=black]{-methodA()};
\node at (171pt,83.8047pt)[below right,color=black]{-methodB()};
\draw[pstyle0] (44pt,180pt) rectangle (186.4632pt,228pt);
\draw[pstyle1] (59pt,196pt) ellipse (11pt and 11pt);
\node at (59pt,196pt)[]{\textbf{\Large C}};
\node at (73pt,189.0156pt)[below right,color=black]{ConcreteState1};
\draw[pstyle2] (45pt,212pt) -- (185.4632pt,212pt);
\draw[pstyle2] (45pt,220pt) -- (185.4632pt,220pt);
\draw[pstyle0] (221pt,180pt) rectangle (363.4632pt,228pt);
\draw[pstyle1] (236pt,196pt) ellipse (11pt and 11pt);
\node at (236pt,196pt)[]{\textbf{\Large C}};
\node at (250pt,189.0156pt)[below right,color=black]{ConcreteState2};
\draw[pstyle2] (222pt,212pt) -- (362.4632pt,212pt);
\draw[pstyle2] (222pt,220pt) -- (362.4632pt,220pt);
\draw[pstyle4] (143.407pt,64pt) ..controls (146.1887pt,64pt) and (148.9704pt,64pt) .. (151.7521pt,64pt);
\draw[pstyle5] (164.7705pt,64pt) -- (155.7705pt,60pt) -- (159.7705pt,64pt) -- (155.7705pt,68pt) -- (164.7705pt,64pt) -- cycle;
\draw[pstyle4] (159.7705pt,64pt) -- (151.7705pt,63.9999pt);
\draw[pstyle5] (130.1562pt,64pt) -- (136.1562pt,68pt) -- (142.1562pt,64pt) -- (136.1562pt,60pt) -- (130.1562pt,64pt) -- cycle;
\draw[pstyle4] (169.6379pt,118.0528pt) ..controls (156.0476pt,139.4308pt) and (141.1249pt,162.9046pt) .. (130.3139pt,179.9107pt);
\draw[pstyle4] (163.7462pt,114.2727pt) -- (180.3833pt,101.1499pt) -- (175.5609pt,121.7835pt) -- (163.7462pt,114.2727pt) -- cycle;
\draw[pstyle4] (238.1767pt,118.372pt) ..controls (251.5603pt,139.6642pt) and (266.2218pt,162.9892pt) .. (276.8582pt,179.9107pt);
\draw[pstyle4] (232.0684pt,121.8078pt) -- (227.3513pt,101.1499pt) -- (243.9213pt,114.3574pt) -- (232.0684pt,121.8078pt) -- cycle;
\end{tikzpicture}
\end{document}
}
\end{figure}
\begin{intentbox}[Intent]\nospacing
    Context related behavior depending on the current state
\end{intentbox}
\begin{partbox}[Participants]\nospacing
  \begin{itemizenosep}
      \item \imp{Context}: is the concrete thing (e.g.\ Vending Machine, Ventilator) that has a reference/stores the concrete state
      and acts depending on it by calling the \javainline{request} or \javainline{doAction} method of its current state reference.
      \item \imp{Abstract/Interface State}: is
    \begin{itemizenosep}
        \item Is an interface that will be implemented by the \textit{concrete states} and defines all the possible actions
        that each \textit{concrete state} may be able to perform or
          \item An abstract base class if
        \begin{itemize}
            \item we want to define default behaviors of the shared methods:
             \begin{mintlinebox}{java}
               privat void doAction1(|\optc{args}|){
                 throw new IllegalStateException();
               }
             \end{mintlinebox}
            \item the concrete states store shared similar attributes/methods that we do not want to define inside the context.
        \end{itemize}
    \end{itemizenosep}
    \item \imp{Concrete State}: Implements the methods of the \textit{state interface}
  \end{itemizenosep}
\end{partbox}
\begin{notebox}[Note]\nospacing
  \begin{itemizenosep}
      \item Contexts may hold multiple different states e.g.\ light state as well as temperature state.
      \item The possible actions declared inside the state interface/abstract class \ul{may take a context object} as argument
      in order to change the current state, if they are not defined inside the context (see \tc{section}{State changes} section).
      \item Interfaces allow only public methods before Java9, may be another reason for using abstract classes.
  \end{itemizenosep}
\end{notebox}
\begin{codeboxNl}[Context]{java}
  class Context{
    State state;
    
    public void setState(State s){
      this.state = s;
    }

    public State getState(){
      return state;
    }

    public doAction1(){
      state.doAction1(|\ul{\opta{this}}|);
    }

  }
\end{codeboxNl}
\begin{codeboxNl}[State]{java}
  abstract class State {
    privat void doAction1(|\ul{\opta{Context cur}}|,|\optc{args}|){
      throw new IllegalStateException();
    }
  }
}
\end{codeboxNl}
\subsection*{State Changes}
\subsubsection{Decentralized}
\begin{sectionbox}\nospacing
  May be initiated by the \rd{concrete state} method $\Rightarrow$
  \begin{itemizenosep}
    \item Concrete state object needs a reference to the \textit{context} in order to set its state:
  \end{itemizenosep}
\end{sectionbox}
  \begin{codeboxNl}[Context]{java}
    class Context{
      State state;
      |\optldots|
      public doActions(){
        state.doAction1(this);
        state.doAction2(this);
      }

    }
  \end{codeboxNl}
  \begin{codeboxNl}[State]{java}
    interface State {
      public |\optc{retType}| doAction1(Context cur,|\optc{args}|);
      public |\optc{retType}| doAction2(Context cur,|\optc{args}|);
    }
  }
  \end{codeboxNl}
  \begin{codeboxNl}[\rd{Concrete State}]{java}
    concStateA implementes State {
      public state doAction1(Context cur, |\optc{args}|);
          // do somthing
          cur.setState(new StateB());
    }
  }
  \end{codeboxNl}
\begin{sectionbox}\nospacing
 \begin{itemizenosep}
    \item \imp{Or} the state returns the new state when running an action which gets then set by the context
 \end{itemizenosep} 
\end{sectionbox}
\begin{codeboxNl}[Context]{java}
  class Context{
    State s;
    |\optldots|
    public doActions(){
      s = state.doAction1();
      s = state.doAction2();
    }
  }
\end{codeboxNl}
\begin{sectionbox}\nospacing
 \begin{itemizenosep}
     \item \imp{Or} the state and Concrete States are defined inside context class, thus concrete
      states can directly set state of the context:
 \end{itemizenosep} 
\end{sectionbox}
\begin{codeboxNl}[\rd{Concrete State}]{java}
class Context{
  State s;
  |\optldots|
  concStateA implementes State {
    public state doAction1(|\optc{args}|);
        // do somthing
        s = setState(new StateB());
  }
  |\optldots|
}
\end{codeboxNl}
\subsubsection{Centralized}
\begin{sectionbox}\nospacing
  Transitions are initiated by the context, state should be informed that it gets activated or deactivated. 
    \begin{codeboxNl}[DrawTool (State)]{java}
      public interface DrawTool{
        void activate();
        void deactivate();
      }
    \end{codeboxNl}
    \begin{codeboxNl}[DrawView (Context)]{java}
      DrawTool tool; // current state
      
      public void setTool(DrawTool tool){ // set conc state;
        if(tool == null) throw new IllegalArgumentException();
        if(this.tool != null) this.tool.deactivate();
        this.tool = tool;
        this.tool.activate();
      }
    \end{codeboxNl}
\end{sectionbox}
\begin{notebox}[Note]\nospacing
  If the initial state is set in the constructor of the context we can omit:
  \javainline{if(this.tool != null) this.tool.deactivate();}
\end{notebox}
\subsection*{Creation of State Objects}
\subsubsection{States are created on demand}
\begin{sectionbox}\nospacing
 \begin{codebox}{java}
   public void stateMethod1(Context cur){
     cur.setState(new StateB());
   }
 \end{codebox} 
\end{sectionbox}
\subsubsection{States are stored}
\begin{sectionbox}\nospacing
  States are stored in variable or list that is accessible by all concrete states
  $\Rightarrow$ thus states are for example stored inside context.
\end{sectionbox}
 \begin{codebox}[Context]{java}
   class Context{
     State s;
     |\optldots|
     private final State CONC_STATE_A = new CStateA();
     private final State CONC_STATE_B = new CStateB();
     private final State INIT = new InitState();
 }
 \end{codebox} 
 \begin{codebox}[Abstract State Class]{java}
   public void stateMethod1(Context cur){
     cur.setState(c.INIT);
   }
 \end{codebox} 
 \begin{codebox}[Abstract State Class]{java}
   public void stateMethod2(Context cur){
     throw new IllegalStateException();
   }
 \end{codebox} 
\begin{notebox}[Examples]\nospacing
  \begin{itemizenosep}
      \item DrawTool (drawing state)
      \item DrawGrid (constrain the mouse coordinates)
      \item Handles (CTRL/SHIFT pressed)
  \end{itemizenosep}
\end{notebox}
%%% Local Variables:
%%% mode: latex
%%% TeX-master: "../formulary"
%%% End:
